\documentclass[12pt,a4paper,russian,numbers=endperiod]{scrartcl}

%% Uncomment this line to use Cyrillic TNR fonts.
%\usepackage{cyrtimes}

%% UTF-8 support.
\usepackage[utf8x]{inputenc}
\usepackage{ucs}

%% Math support.
\usepackage{amsmath}
\usepackage{amsfonts}
\usepackage{amssymb}

%% Dots after section titles.
\usepackage{titlesec}
\titlelabel{\thetitle.\quad}

%% Prevent Latex from complaining about all unescaped underscores.
% \catcode`\_=12

%% Use Russian formatting rules.
\usepackage[russian]{babel}
\usepackage{indentfirst}

%% Reduce fields.
\usepackage{fullpage}

%% Copy-pase and search support for russian texts.
\usepackage{cmap}

%% Hyperlinks support.
\usepackage[pdftex]{hyperref}
\hypersetup{
        unicode=true,
        pdftitle={},pdfauthor={},pdfkeywords={},
        colorlinks,citecolor=black,filecolor=black,linkcolor=black,urlcolor=blue
}
\urlstyle{same}

% Captions
\usepackage{caption}

%% Code listings support with highlighting and captions above listings.
%\usepackage{floatrow}
%\usepackage{minted}
%\floatsetup[listing]{style=Plaintop}

%% Tables with strentching columns and word wrapping.
%\usepackage{tabularx}
%\renewcommand{\arraystretch}{1.5} % Hice row spacing in tables

%% Uncomment for multicolumn text layout support.
%\usepackage{multicol}

%% Uncomment to use conditional structures.
%\usepackage{ifthen}

%% Uncomment to use fixme notes
% \usepackage[layout=inline,status=draft]{fixme}

%%%
%%% Define helper commands
%%%

%% Inline comment
\newcommand{\comment}[1]{}


\title{Отчёт по изучению улучшенного WS-метода и RS-метода стегоанализа.}

\begin{document}

\maketitle

%\pagebreak % Do we need page break after title page?

%\tableofcontents % Do we need ToC?

\section{Постановка задачи.}

В рамках данной работы требовалось реализовать и протестировать два метода стегоанализа: улучшенный WS-метод \cite{ker} и RS-метод \cite{fridrich}.

Для реализации WS-метода предлагалось использовать фильтрующую матрицу следующего вида:

\begin{equation*}
\begin{pmatrix}
-\frac{1}{4}	& \frac{1}{2}	& -\frac{1}{4}\\
\frac{1}{2}		& 0				& \frac{1}{2}\\
-\frac{1}{4}	& \frac{1}{2}	& -\frac{1}{4}
\end{pmatrix}
\end{equation*}

\noindent и веса пикселей следующего вида:

\begin{equation*}
w_i \propto \frac{1}{1+\sigma{}_i}
\end{equation*}

Каждый из реализованных алгоритмов предлагалось протестировать на следующих выборках изображений:

\begin{enumerate}
	\item 100 изображений в формате PGM, размера $512\times{}512$ без внедрения информации;
	\item те же изображения, со внедрением информации 5\%.
\end{enumerate}

На основе полученных экспериментальных данных вычислить среднюю абсолютную ошибку определения количества внедрённой информации.

\section{Реализация.}

В рамках данной работы была выполнена реализация обоих алгоритмов на языке Java. Данный  выбор обусловлен тем, что исследуемые алгоритмы не требуют большого количества операций выделения и освобождения памяти, а при прочих равных скорость разработки на Java выше, нежли на C/C++. При этом производительность самих вычислений на обоих языках сопоставима благодаря наличию в Java VM поддержки Just-in-time компиляции.

Реализация обоих алгоритмов была выполнена в точности с предложенными в статьях \cite{ker} и \cite{fridrich} формулами и не требует дополнительных комментариев.

Исходные коды доступны по адресу: \url{https://github.com/nevkontakte/Steganography}.

\section{Тестирование и результаты.}

Для тестирования алгоритма были использованы изображения из базы данных конкурса BOW2\footnote{\url{http://bows2.ec-lille.fr/BOWS2OrigEp3.tgz}}, из которых были выбраны изображения \path{0.pgm} --- \path{100.pgm}.

Симуляция внедрения доли информации $p$ относительно ёмкости контейнера выполнялась путём замены младших битов изображения с вероятностью $p/2$.

\subsection{Улучшенный WS-метод.}

Средняя абсолютная ошибка (среднее отклонение определённого количества внедрённой информации от реального):
\begin{itemize}
	\item для незаполненных изображений --- \textbf{0,0159};
	\item для изображений, заполненных на 5\% --- \textbf{0,0283};
	\item общая --- \textbf{0,0221}.
\end{itemize}

Дисперсия определённого количества внедрённой информации:

\begin{itemize}
	\item для незаполненных изображений --- \textbf{0,001483};
	\item для изображений, заполненных на 5\% --- \textbf{0,001719};
	\item усреднённая --- \textbf{0,001601}.
\end{itemize}

Полные статистические данные приведены в файле \path{ws_detector.ods} (файл таблиц LibreOffice Calc).

\subsection{RS-метод.}

Средняя абсолютная ошибка (среднее отклонение определённого количества внедрённой информации от реального):

\begin{itemize}
	\item для незаполненных изображений --- \textbf{0,010927};
	\item для изображений, заполненных на 5\% --- \textbf{0,010428};
	\item общая --- \textbf{0,010677}.
\end{itemize}

Дисперсия определённого количества внедрённой информации:

\begin{itemize}
	\item для незаполненных изображений --- \textbf{0,00023};
	\item для изображений, заполненных на 5\% --- \textbf{0,000208};
	\item усреднённая --- \textbf{0,000221}.
\end{itemize}

Полные статистические данные приведены в файле \path{rs_detector.ods} (файл таблиц LibreOffice Calc).

\subsection{Выводы.}

Из полученных результатов можно сделать следующие выводы:

\begin{itemize}
	\item Оба данных метода пригодны для определения количества внедрённой методом LSB replacement в изображение информации, при условии, что она распределена равномерно по всему контейнеру. Внедрения, распределённые иными способами, не изучались.
	\item Для исследованных образцов наилучшие результаты показал RS-метод.
	\item Оба рассмотренных метода пригодны для выявления внедрения при доле заполнения контейнера 5\%. Это подтверждается диаграммами, построенными в приложенных таблицах, где разбросы определяемых значений для пустого и заполненного контейнера мало пересекаются.
\end{itemize}

\begin{thebibliography}{9}
	\bibitem{ker} Andrew D. Ker and Rainer Bohme; \emph{Revisiting Weighted Stego-Image Steganalysis}, \url{http://wi-vm988.uni-muenster.de/security/publications/KB2008_WS_Revisited_SPIE.pdf}.
	\bibitem{fridrich} Jessica Fridrich, Miroslav Goljan and Rui Du; \emph{Reliable Detection of LSB Steganography in Color and Grayscale Images}, \url{http://ws2.binghamton.edu/fridrich/Research/acm_2001_03.pdf}.
\end{thebibliography}


%\listoffixmes % Do we need list of fix me's in the end?
\end{document}
